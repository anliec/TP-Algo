\documentclass[a4paper, 12pts]{article}

\usepackage[top=3.5cm, bottom=3.5cm, left=3cm, right=3cm]{geometry}

\usepackage[T1]{fontenc}
\usepackage[utf8]{inputenc}
\usepackage[francais]{babel}
\usepackage{textcomp}
\usepackage{listings}
\lstset{language=C++} 

%\usepackage{hyperref} %pour les liens internet

%\usepackage{graphicx} %pour les images
\title{TP2 C++}
\author{B3111: Edern HAUMONT et Nicolas SIX}
\date{Mercredi 11 novembre 2015}

%to define the subsubsubsection structure:
\usepackage{titlesec}
%\usepackage{hyperref}

\titleclass{\subsubsubsection}{straight}[\subsection]

\newcounter{subsubsubsection}[subsubsection]
\renewcommand\thesubsubsubsection{\thesubsubsection.\arabic{subsubsubsection}}
\renewcommand\theparagraph{\thesubsubsubsection.\arabic{paragraph}} % optional; useful if paragraphs are to be numbered

\titleformat{\subsubsubsection}
  {\normalfont\normalsize\bfseries}{\thesubsubsubsection}{1em}{}
\titlespacing*{\subsubsubsection}
{0pt}{3.25ex plus 1ex minus .2ex}{1.5ex plus .2ex}

\makeatletter
\renewcommand\paragraph{\@startsection{paragraph}{5}{\z@}%
  {3.25ex \@plus1ex \@minus.2ex}%
  {-1em}%
  {\normalfont\normalsize\bfseries}}
\renewcommand\subparagraph{\@startsection{subparagraph}{6}{\parindent}%
  {3.25ex \@plus1ex \@minus .2ex}%
  {-1em}%
  {\normalfont\normalsize\bfseries}}
\def\toclevel@subsubsubsection{4}
\def\toclevel@paragraph{5}
\def\toclevel@paragraph{6}
\def\l@subsubsubsection{\@dottedtocline{4}{7em}{4em}}
\def\l@paragraph{\@dottedtocline{5}{10em}{5em}}
\def\l@subparagraph{\@dottedtocline{6}{14em}{6em}}
\makeatother

\setcounter{secnumdepth}{4}
\setcounter{tocdepth}{4}




%-----------------------------------------------------------------------------------------


\begin{document}

\begin{titlepage}

\maketitle

\end{titlepage}

%----------------------------------------------Title end

\tableofcontents

\pagebreak

%----------------------------------------------table of contents end

%==================================================================================================
%                                     Section: Introduction
%==================================================================================================

\section{Introduction}
\paragraph{}

\subsection{Choice of the data structure}
\paragraph{}
Our application must support 20 million events from 1500 sensors. Consequently, we chose not to store event themselves as they arrive. However, at the launch of our program, several static arrays of entire values are created. Their indexes correspond to the information used to anwer to the user querries, for instance the sensor number, the hour,etc. Their last dimension corresponds to a sensor color. So when an event is added, we just increment one cell in each array.

%==================================================================================================
%                                     Section: Clases
%==================================================================================================
\section{Classes}


%<<<<<<<<<<<<<<<<<<<<<<<<<<<<<<<<<<<<<<<<<<<<<<<<<<<<<<<<<<<<<<<<<<<<<<<<<<<<<<<<<<<<<<<<<<<<<<
%                                   DataHandler
%<<<<<<<<<<<<<<<<<<<<<<<<<<<<<<<<<<<<<<<<<<<<<<<<<<<<<<<<<<<<<<<<<<<<<<<<<<<<<<<<<<<<<<<<<<<<<<
\subsection{DataHandler}
\paragraph{}
All data additions and request are managed by the class DataHandler which contains itself the 4 data arrays.

%----------------------------------------------------------------------------------------
%                                  Constants
%----------------------------------------------------------------------------------------
\subsubsection{Constants}

\paragraph{}
	This class use differents constants, first there is one error code used when the traffic char is not one of the four expected: 'V','J','R','N'

\begin{lstlisting}
const unsigned int ERROR_INVALID_TRAFIC_UCHAR = 201;
\end{lstlisting}

\paragraph{}
	DataHandler also uses constants to define the size of the arrays used to store all the data.

\begin{lstlisting}
const int NUMBER_OF_COLORS = 4;
const int NUMBER_OF_SENSORS = 1500;
const int NUMBER_OF_MINUTES = 1440;
const int NUMBER_OF_HOURS = 24;
const int NUMBER_OF_DAYS = 7;
\end{lstlisting}

%----------------------------------------------------------------------------------------
%                                  Atributes
%----------------------------------------------------------------------------------------
\subsubsection{Atributes}
\paragraph{}
	To make our code more readable we decided to use the following type def in the DataHandler Class.
\begin{lstlisting}
typedef unsigned int uint;
typedef unsigned short int usint;
typedef unsigned char uchar;
\end{lstlisting}

\paragraph{}
	The atributes used in this class are the following:
\begin{lstlisting}
IdHash idHash;
uint sensors[NUMBER_OF_SENSORS][NUMBER_OF_COLORS];
uint days[NUMBER_OF_DAYS][NUMBER_OF_COLORS];
uint daysAndHours[NUMBER_OF_DAYS][NUMBER_OF_HOURS][NUMBER_OF_COLORS];
uchar *daysAndMin[NUMBER_OF_DAYS][NUMBER_OF_MINUTES][NUMBER_OF_COLORS];
\end{lstlisting}

\paragraph{}
	idHash is an instance of our IdHash class, which is detailled latter. It's an hash tab that we use to link the sensors id and the position in our static tabs.
	sensors, days, daysAndHours and daysAndMin are four static array used to store statistics which will be used in our stats methods. The fourth array is a four dimension array. The last dimension coreponds to the sensors id and is alocated using a new in the construtor.

%----------------------------------------------------------------------------------------
%                                  Public Methods
%----------------------------------------------------------------------------------------
\subsubsection{Public Methods}

\subsubsubsection{addData}
\begin{lstlisting}
int addData(const char &trafic,const uint &min,const uint &hours,\
	const uint &id,const uint &day7);
\end{lstlisting}
\paragraph{description:}
	Method used to update member arrays in corresponding cells. Complexity: O(1).
\paragraph{Contract:}
	The values given must be rightly built: min should be between 0 and 59, hours between 0 and 23 and day7 between 0, for monday, and 6, for sunday. To work properly traffic must be set to one of: 'V','J','R','N'.

\subsubsubsection{sensorStats}
\begin{lstlisting}
int sensorStats(uint id) const;
\end{lstlisting}
\paragraph{description:}
	Prints the sensor statistics for an id. Complexity: O(1).
\paragraph{Contract:}
	The id given in parameter must corespond to a sensor with an already added id. Otherwise the stats will use the first added sensor. Prints null stats if 0 corresponding event

\subsubsubsection{jamStats}
\begin{lstlisting}
int jamStats(uchar day7);
\end{lstlisting}
\paragraph{description:}
	Prints jam statistics for a day. Complexity: O(1).
\paragraph{Contract:}
	day7 must be between 0 and 6 $(NUMBER\_OF\_DAYS-1)$. Prints null stats if 0 corresponding event.
		
\subsubsubsection{dayStats}
\begin{lstlisting}
int dayStats(uchar day7);
\end{lstlisting}
\paragraph{description:}
	prints a week day statistics for a given day. Complexity: O(1).
\paragraph{Contract:}
	day7 must be between 0 and 6 $(NUMBER\_OF\_DAYS-1)$. Prints null stats if 0 corresponding event
		
\subsubsubsection{optimum}
\begin{lstlisting}
int optimum(uchar day7, uint begginHours, uint endHours,\
		uint idTab[], uint tabSize);
\end{lstlisting}
\paragraph{description:}
	optimum looks for the best departure time in a given interval. Complexity: $O(tabsize * (endHours-beginHours) )$.
\paragraph{Contract:}
	day7 must be between 0 and 6 $(NUMBER\_OF\_DAYS-1)$. begginHours and endHours must be between 0 and 23 with begginHours smaller than endHours. idTab must be an array of size tabSize filled with already added sensors' id (if an id is not added the first added sensors will be used).
	
	To compute the optimum departure time, this methode considers that the time taken to go through a sensor is the most probable one at this time of the day according to the added data. In case of equality on trafic information, the quicker will be used (optimistic).

%<<<<<<<<<<<<<<<<<<<<<<<<<<<<<<<<<<<<<<<<<<<<<<<<<<<<<<<<<<<<<<<<<<<<<<<<<<<<<<<<<<<<<<<<<<<<<<
%                                   IdHash
%<<<<<<<<<<<<<<<<<<<<<<<<<<<<<<<<<<<<<<<<<<<<<<<<<<<<<<<<<<<<<<<<<<<<<<<<<<<<<<<<<<<<<<<<<<<<<<
\subsection{IdHash}

\paragraph{}
This class makes the link between a sensor id of the user and indexes in data tables
It is a hashTable and its structure is an array.
It is not designed for more than 25000 ids.

%----------------------------------------------------------------------------------------
%                                  Constants
%----------------------------------------------------------------------------------------
\subsubsection{Constants}

\paragraph{}
	This Class use differents constants, most of them are related to the hash function.

\begin{lstlisting}
const int SIZE_OF_HASHTABLE = 5000;
//hash function parameters
const int PRIME_NUMBER = 7001;
const int A = 1;
const int B = 0;
\end{lstlisting}

%----------------------------------------------------------------------------------------
%                                  Atributes
%----------------------------------------------------------------------------------------
\subsubsection{Atributes}

\paragraph{}
	The private atributes used in this class are the following:
\begin{lstlisting}
unsigned * hashTable[SIZE_OF_HASHTABLE][2];
unsigned sizeOfHashList[SIZE_OF_HASHTABLE][2];
unsigned lastIdInTab;
\end{lstlisting}
\paragraph{}	
    hashTable is an array of two dimensions plus one last dynamic one. The first one is the hash array itself and corresponds to the position given by the hash function. The second one is there to have both the sensors id and the position on the program array. The last dimention is a way to handel case where the hash function give the same position to tow different id.
    
    sizeOfHashList is an two dimensions array created to now the size and the used size of each dynamic part of hashTable.
    
    lastIdInTab keeps the value of the last added position in order to just have to increment it for the next add.
    
    
%----------------------------------------------------------------------------------------
%                                  Public Methods
%----------------------------------------------------------------------------------------
\subsubsection{Public Methods}

\subsubsubsection{addId}
\begin{lstlisting}
unsigned addId(const unsigned & id);
\end{lstlisting}
\paragraph{description:}
	add the user id to the hash table if not already in and returns an array position used by the program.
\paragraph{Contract:}
	id in int range.
\subsubsubsection{getTabId}
\begin{lstlisting}
unsigned getTabId(const unsigned & id) const;
\end{lstlisting}
\paragraph{description:}
	returns the array position used by the program corresponding to a sensor id called by the the user.
\paragraph{Contract:}
	id in unsigned int range. If id is not in the hash tab, returns 0.
	
%----------------------------------------------------------------------------------------
%                                  Hash function
%----------------------------------------------------------------------------------------
\subsubsection{Hash function}

\paragraph{}
	In order to have a good repartition of our datas on the hash tab we use the classical:
\begin{equation}
	hash(id)=(A*id+B)\%PRIME\_NUMBER)\%SIZE\_OF\_HASHTABLE
\end{equation}	
	With $A$, $B$, $PRIME\_NUMBER$ and $SIZE\_OF\_HASHTABLE$ the constants defined in the constents section.
	
	This function can be improved with random $A$ and $B$ generated at the begining of the execution in order to forgive a data set to always be the worst one.

%==================================================================================================
%                                     Section: tests
%==================================================================================================
%==================================================================================================
%                                     Section: tests
%==================================================================================================
\section{tests}

\paragraph{Introduction}
We used a few manually generated test and we generated some bigger input files with a generation code.
This script also ofently generates an expected output file.

To test or application, we made it read th input file instead of the keyboard input. We also redirect the application output to another file which is afterwards compared to the expected output file.

\subsection{manually generated tests}

\subsubsection{TestAdd}
\paragraph
The first test just check that the ADD query runs fine. We add several event and check each time the state of the table with STATS_C.

\subsubsection{TestAddStatC and TestStatC}
\paragraph
These tests check the comportment of the STATS_C query, especially with no values, one value, and after several ADD.

\subsubsection{TestStatD7}
\paragraph
This test checks the comportment of the STATS_D7 query, especially with no values, one value, and after several ADD.

\subsubsection{TestJamDH}
\paragraph
This test checks the comportment of the JAM_DH query, especially with no values, full jam values and multiple values.

\subsubsection{TestOpt}
\paragrap
This test checks the comportment of the OPT query. We made this request with 0 sensors, sensors with no value, 1 and 2 sensors.


\subsection{machine generated tests}

\subsubsection{Test2}
\paragrap
The code generates entries (all are green) for 1500 sensors and a restricted time interval. Calls JAM_DH query to test input values. Used to check that the program runs.

\subsubsection{Test3}
\paragrap
The code generates entries for 50 sensors but in bigger time intervals and multiple traffics. Calls the 3 different queries with several parameters. Used o check that the program computes datas properly.

\subsubsection{Test4}
\paragrap
The code generates some entries to test basically if the last query works with some missing values. Fast to check.

\subsubsection{Test5}
\paragrap
Our biggest test. ~20M events on 1500 sensors. To compare with outputs from other groups. Sensor numbers between 0 and 1G



\end{document}





